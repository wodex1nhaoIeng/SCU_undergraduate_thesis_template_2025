%--------------------------------------------------------------------
%导言区
% !Mode:: "TeX:UTF-8"

\documentclass[a4paper,UTF8, AutoFakeBold]{ctexrep}

\usepackage{scuthesis}				% 封面版式
\usepackage{amssymb}  				% 设置数学公式
\usepackage{amsmath} 				% 设置数学公式编号
\usepackage{amsthm}					% 定理

%\usepackage[all,cmtip]{xy}			% xy-pic	画交换图
\usepackage{tikz}			        % tikz      绘图宏包
\usepackage{float}	  				% float		为固定图片位置宏包
\usepackage{subfigure}				% subfigure 引入宏包来添加多张图片 
\usepackage{caption}				% caption   为更改图片命名的宏包
\usepackage{enumerate}				% enumerate 有序列表环境
\usepackage{ctex}

%\usepackage{boondox-cal}			% boondox-cal   数学花体
%\usepackage{bm}       				% bm 			希腊字母加粗(普通字母类同)

\theoremstyle{plain}
	\newtheorem{thm}{定理~}[chapter]
	\newtheorem{lem}[thm]{引理~}
	\newtheorem{prop}[thm]{命题~}
	\newtheorem{cor}[thm]{推论~}
\theoremstyle{definition}
	\newtheorem{defn}[thm]{定义~}
	\newtheorem{conj}[thm]{猜想~}
	\newtheorem{exmp}[thm]{例~}
	\newtheorem{ques}[thm]{问题~}
	\newtheorem{rem}[thm]{注~}

%	该命令指定公式编号的格式
\numberwithin{equation}{chapter}
\renewcommand{\theequation}{\thechapter.\roman{equation}}

% 请在此处添加或修改你想要的定理样式,以下为英文定理样式,若使用中文写作请注释以下部分,并改用上面的中文定理样式(注意,使用英文写作时,若完成该操作后仍有部分单词显示为中文,请参照ctex文档第6节的“文档汉化”部分自行调整):
% \theoremstyle{plain}
%   \newtheorem{thm}{Theorem}[chapter]
%   \newtheorem{lem}[thm]{Lemma}
%   \newtheorem{prop}[thm]{Proposition}
%   \newtheorem{cor}[thm]{Corollary}
% \theoremstyle{definition}
%   \newtheorem{defn}[thm]{Definition}
%   \newtheorem{conj}[thm]{Conjecture}
%   \newtheorem{exmp}[thm]{Example}
%   \newtheorem{ques}[thm]{Question}
%   \newtheorem{rem}[thm]{Remark}
% \ctexset{bibname = {References}}
% \ctexset{proofname = {Proof}}
% \ctexset{contentsname = {Contents}}

%------------------------------------------------
	%基本信息
	\title{四川大学本科毕业论文模板(非官方)}
	\titleEng{SCU Undergraduate Thesis Template(unofficial)SCU Undergraduate Thesis Template}
	
	\author{孙悟空}
	\authorEng{Wukong Sun}
	\adviser{唐三藏}
	\adviserEng{Sanzang Tang}
	
	\college{计算机学院}
	\collegeEng{School of Computer Science}
	\major{计算机科学与技术}
	\majorEng{Computer Science}
	
	\grade{20xx}
	\id{20xxxxxxxxxxx}
	\date{\today}
	
 			% 作者信息


%-----------------------------------------------------------------
%正文区
	\begin{document}
	\zihao{-4}
	
	% 封面+摘要
	\makecover

\begin{abstract}{\LaTeX; 本科; 毕业论文; 模板}
四川大学本科毕业论文(设计)模板。
\end{abstract}


\begin{abstractEng}{\LaTeX; Undergraduate; Thesis; Template}
LaTex thesis template is for SiChuan University students.The template is 
based on the original version, which was established by \textit{Flying of Death} in 2013. I have changed some places to fit in newly demand
(relevant document in 2018). Please note that it is an unofficial 
template. 
\end{abstractEng}


\tableofcontents
	
	
	% 正文
	\include{src/chap01}
	\include{src/chap02}
	
	
	% 后记(附录)
	\include{src/epilogue}
	
	
	% 参考文献
	\cleardoublepage
	\addcontentsline{toc}{chapter}{参考文献}		% 在目录中添加参考文献
	\bibliographystyle{unsrt}
	\bibliography{ref/refs}
	
	
	% 声明
	\clearpage
	\addcontentsline{toc}{chapter}{声\hspace{0.8cm}明}
	%-----------------------------------------------------------------------
%declaration.tex

	\chapter*{声\hspace{0.8cm}明}


%-------------------------------------------------------------------
	本人声明所呈交的学位论文是本人在导师指导下进行的研究工作及取得的研究成果。据我所知,除了文中特别加以标注和致谢的地方外,论文中不包含其他人已经发表或撰写过的研究成果,也不包含为获得四川大学或其他教育机构的学位或证书而使用过的材料。与我一同工作的同志对本研究所做的任何贡献均已在论文中作了明确的说明并表示谢意。
	
	本学位论文成果是本人在四川大学读书期间在导师指导下取得的,论文成果归四川大学所有,特此声明。
	
	\vspace{40pt}
	\begin{flushright}
		\begin{tabular}{b{4cm} >{\centering\arraybackslash}b{2.5cm} }
			\songti \zihao{-4} 学位论文作者(签名)& {} \\[-3pt] 
			\cline{2-2} \\ [0.6cm]
			\songti \zihao{-4} 论文指导教师(签名)& {} \\[-3pt] 
			\cline{2-2} \\ [0.6cm]
		\end{tabular}
		
		\today
	\end{flushright}
	
	

	% 致谢
	\clearpage
	\addcontentsline{toc}{chapter}{致\hspace{0.8cm}谢}
	\include{src/acknowledgement}

	
	
\end{document} 
	